\documentclass[a4paper,10pt]{article}

\usepackage{mathptmx}
\usepackage{epsf}           %\input{epsf}
    %\usepackage{amsfonts}
%\usepackage{amstext}
%\usepackage{amssymb}
%\usepackage{euscript}
\usepackage{url}
\usepackage[dvips]{graphics}
\usepackage[dvips,all]{xy}
%\usepackage[round,authoryear]{natbib}
\usepackage{multicol}

%\newlength{\extraplusheight}
%\newlength{\extrapluswidth}
%\setlength{\extraplusheight}{4.7cm}
%\setlength{\extrapluswidth}{4.7cm}
%\addtolength{\textwidth}{\extrapluswidth}
%\addtolength{\textheight}{\extraplusheight}
%\addtolength{\oddsidemargin}{-.5\extrapluswidth}
%\addtolength{\evensidemargin}{-.5\extrapluswidth}
%\addtolength{\topmargin}{-0.5\extraplusheight}
\setlength{\parindent}{0.1in}
\setlength{\parskip}{0.4ex}


\setlength{\topmargin}{-2cm}
\setlength{\textheight}{26cm}
\setlength{\oddsidemargin}{-0.68cm}
\setlength{\textwidth}{17.1cm}




% Discourage unnecessary hyphenation.
\sloppy\hyphenpenalty 4000

% Used to mark up query in the text.
\newcommand{\marginalia}[1]{\mbox{}\marginpar[\hfill#1]{#1}}
\newcommand{\query}[2][?]{%
  \marginalia{\parbox{\marginparwidth}{%
      \raggedright\fussy\footnotesize#2}}\dag}
\newcommand{\patty}[1]{{\bf patty: {#1}}}
\newcommand{\ondrej}[1]{ \begin{quote}{\it{\bf Ondrej:}{#1}}\end{quote}}
\newcommand{\vague}[1]{[\emph{#1}]}
%\newcommand{\patty}[1]{}

\newcommand{\gap}{\protect{\hspace*{0.5em}}}

\begin{document}

\thispagestyle{plain}
\begin{center}
  {\Large {\bf Theory and Applications of Higher Dimensional Category Theory}}\\ {\Large \bf Case for Support}\\[1ex]
  \rule{140mm}{.5mm}\\[2ex]
\end{center}

\vspace*{-0.2in}

\subsection*{Part 1A: Previous Research and Track Record - Prof Neil Ghani}
See also \url{http://www.cis.strath.ac.uk/~ng/}

Prof. Ghani obtained his PhD in Computer Science from the University
of Edinburgh in 1995 where he worked on categorical models of
rewriting.  Subsequently he received grants from the Royal Society of
London and the EU-funded EUROFOCS program to conduct post-doctoral
research at the Ecole Normale Superieure in Paris. In January 1997, he
became a Research Fellow at the University of Birmingham and then in
September 1998 he became a Lecturer in the Department of Mathematics
and Computer Science at the University of Leicester. In September
2005, he moved to the University of Nottingham where he became a
Reader in January 2007. On July 1 2008, he was offered a professorship
at the University of Strathclyde and asked to form the Mathematically
Structured Programming research group there.

{\bf Research Summary:} Prof. Ghani's research tries to understand the
nature and structure of computation. This is quite a bold statement
and inevitably, only partial answers will be
forthcoming. Nevertheless, this statement shows his
commitment to ask deep and fundamental questions so as to produce
research which is of the highest calibre and which will stand the test
of time rather than become obsolete within a few years. In particular,
he has worked extensively in the following areas which form the
pillars upon which this proposal is built.

\begin{itemize} \item {\bf Category Theory:} Category theory is a
  relatively new mathematical discipline which provides an abstract
  theory of structure and hence is key to Prof. Ghani's work. He has applied
  various categorical structures such as monads, comonads, coalgebras,
  enriched categories and Kan extensions to problems in computation.
  Of closest relevance to this proposal is his work on containers which
  provides a theory of concrete data types.

\item {\bf Type Theory:} Prof. Ghani uses type theory as an intermediate
abstraction between functional programming and its categorical
underpinnings. He has worked on features such as type systems, pattern
matching and explicit substitutions which make the lambda-calculus
closer to ``real'' functional languages. 
%I was awarded grants from the
%Royal Society of London and from the EUROFOCS programme to conduct
%part of this research. 
He also developed the subject of eta-expansions and showed it to be
better behaved than the more traditional theory of eta-contractions. He
solved the long standing open problem of the decidability of
beta-eta-equality for sum types which had attracted the attention of a
number of research groups across the world.

\item {\bf Programming Languages:} Humans are good at high level,
abstract thinking and programming languages should reflect that so as
to make it easier for humans to program. Functional languages offer
that possibility so Prof. Ghani has been active in their development.  
Of clear relevance to this project is his work on initial algebra
semantics for advanced data types such as nested types and GADTs. He has
also worked on short cut fusion and the problem of how to compose
monads.

% is the corner stone of inductive types. I showed
% that it was, counter to popular belief, expressive enough to define
% canonical forms of recursion which occur with nested types. I also
% extended initial algebra semantics to GADTs. I also extended initial
% algebras with an additional universal property based upon the
% characterisation of initial algebras as limits. This has generated new
% Church encodings for inductive types and placed short cut fusion at
% the centre of initial algebra semantics  
\end{itemize}

{\bf Publications:} Prof. Ghani has written over 45 papers which have been
published in internationally refereed conferences and journals. These
can be found from the above url. Especially relevant are papers on
initial algebra semantics~\cite{ghani-tlca07,ghani-popl08} and
containers~\cite{ghani-fossacs03,ghani-appsem04,ghani-fi04,ghani-icalp04,ghani-jcats07}.

{\bf Grants:} Prof. Ghani has been awarded the following research grants: i)
Theory and Applications of Containers. EPSRC 2005-2008. Principal
Investigator. Grant Value: 230,000 pounds. Code EP/C511964/1; ii)
Midlands Graduate School. EPSRC 2004-2006. Principal
Coalgebra and Recursion. The Royal Society of London, 2003-2005. Sole
Investigator. Grant Value: 12,000 pounds. Code GR/T06087/01; iii)
Investigator. Grant Value: 6,000 pounds; iv) Kan - A Categorical
approach to Computer Algebra. EPSRC 2001 - 2004. Sole
Investigator. Grant Value: 130,000 pounds. Code GR/R29604/01. End of
grant IGR assessment: Tending to Outstanding; v) Categorical
Rewriting: Monads and Modularity. EPSRC 2000 - 2002. Sole
Investigator. Grant Value: 52,000 pounds. Code GR/M96230/01.  End of
grant IGR assessment: Outstanding; and vi) Eta-Expansions in Dependent
Type Theory. EUROFOCS 1996. Sole Investigator. Grant Value: 10,000
pounds.

{\bf Esteem Indicators:} Prof. Ghani is a current member of the EPSRC
college charged with assessing the quality of grant applications. He
was the director of the Midlands Graduate School in the Foundations of
Computer Science. He is also on the steering committee for the British
Colloquium on Theoretical Computer Science. He has succesfully
supervised 2 PhD students, has two more in their third year and has
supervised 2 RAs. He has been the external examiner for 2 PhD
students. This year, he has been on the PC's of PPDP, CMCS and RTA.


\section*{Part 1B: Previous Research Track Record - Dr Eugenia Cheng}

\vspace*{1in}

\section*{Part 1C: Previous Research Track Record - Dr Ondrej Rypacek}


\vspace*{1in}

\pagebreak


\section*{Part 2: The Proposed Research and its Context}

\subsection*{2.1  \gap Introduction}\label{sec:intro}

{\bf What is Higher Dimensional Category Theory:} When are two things
equal?  This is an age old philosophical problem which has caused
confusion throughout the centuaries. 

\begin{itemize}
\item Mathematics is typically extensional
\item Logic/computation/constructivism is intensional
\item definitional equality vs propositonal equality?
\item Both challenge each other. Mathematica foundations of
  computation, comptational support for mathematics
\end{itemize}  
Remarkably, we are finally on the verge of making real and
significant progress on understaning equality.
\begin{itemize}
\item CT is a theory of structure replacing set theory as it is more
  high level. Highly axiomatic. Algebraic complementation to Logic. CT
  moves away from looking into objectsand looks instead at morphisms.
\item HDCT offers to intensionalise the extensional standard CT. It is
  algebraic to complement to the logical perspectives of HDTT.
\item HDTT has come along too.
\end{itemize}

{\bf The Project:} Although the ideas behind higher dimensional
category theory were a fundamental innovation, they have yet to become
widely used. We believe this is because i) the conceptual and
theoretical foundations of higher dimensional category theory need
further development; and ii) more examples and applications of the use
of higher dimensional category theory are needed to illustrate its use
to a broader audience. Therefore we intend to pursue a twin-tracked
research programme based upon a synthesis of theoretical and applied
research:
\begin{itemize}
\item {\bf Theoretical Foundations:} Unify definitions of higher
  dimensonal categories, develop higher dimensional variants of
  functors, natural transformations, adjoints, provide tools to reduce
  the complexity of reasoning in HDCT and run case studies.
\item {\bf Practical Applications:} We will demonstrate the potential
  for higher dimensional category theory to make a fundamental impact
  on both mathematics and computation by developing a number of naturally
  occurring applications. Principally, these applications will be in
  the areas of type theory, rewriting and homotopy theory. These are
  ideal application areas as HDCT not only solves problems that
  researchers currently study, but also suggests new research
  directions, perspectives and ideas within these areas.
\end{itemize}

Taken as a whole, this body of research will provide complementary
algebraic and techniques to address the fundamental and inter-related
questions of i) what can we do with HDCT; and ii) how it can be done
in as clean and cogent a form as possible.

{\bf Calibre and Feasibility:} The fundamental nature of the problem
of what makes mathematical and computational entities equal, and the
consequent potential for applications in a variety of different
fields, attest to the quality and calibre of this project. Our
ambition is demonstrated by our central belief that higher dimensional
category theory can repilcate the ability of standard category theory
to provide a unifying, axiomatic and foundational framework for
mathematics and computer science.

The feasibility of the project is demonstrated by i) the involvement
of the world leaders (Obama, Putin etc) in the project as project
partners; and ii) more generally, our experience in category theory, type
theory and the application of these foundational disciplines to type
theory and programming languages (in particular Agda and Epigram).

\subsection*{2.2 \gap Scientific and Technological Relevance:} 

{\bf Category Theory:} One of the most remarkable accomplishments of
the early 20th century was to formalize all of mathematics in terms of
a language with a deliberately impoverished vocabulary: the language
of set theory. However, set theory's low-level nature is a deep
problem when it comes to its use - we understand the world in terms of
high level-abstractions. As a result category theory, developed in the
second half of the 20th centuary, has gained ever wider traction in
matheamtics, computer science, physics, biology etc because it
provides a high level theory of structure. At a simple level, while
set theory seeks to understand entities as sets and their elements,
category theory treats entities as "black boxes" and instead
understands them via the morphisms between them

{\bf Higher Dimensional Category Theory:}

\begin{itemize}
\item Morphisms between morphisms
\item HDCT is not n-Cat theory
\end{itemize}

{\bf Type Theory:} 
\begin{itemize}
\item Martin L\"of Type Theory
\item Identity types
\item HDTT
\end{itemize}

{\bf Programming Languages:} Abstraction is vital not only in mathematics - it
is equally essential in programming where identifying common structure
is essential to ensure code is clear, clean and concise. Type theory
and category theory have already helped drive the development of such
abstractions, eg dependent types, monadic programming and initial
algebra semantics.

{\bf Rewriting:} Rewriting~\footnote{For the sake of this proposal we
  think of operational semantics as a form of rewriting} forms a
simple model of computation by seeking to understand how one entity
can be replaced by an extensionally equal, but computationally
simpler, entity. In a nutshell, rewriting is an asymetric form of
equality where equality relations are directed to model reduction.

We intend to build upon this convergence of ideas between
mathematicians and computer scientists by demonstrating what is
available to them, what can be done with it, and introducing them to
HDCT definitions, constructions and theorems they are currently
unaware of. And, with advances in dependently typed programming
languages, we can give programmers code to experiment and play
with. This is key to ensuring that HDCT breaks out of the small
community which works on HDCT and becomes as widely used and essential
to researchers as category theory has become.


\subsection*{2.3 \gap Relevance to Beneficiaries}

{\bf Relevance to Beneficiaries:} This, perhaps more than many
projects, is an ambitious project which has the potential to have a
significant impact on a large number of researchers. We believe that
there are now a critical mass of examples of higher dimensional
structures in mathematics, computation, physics and biology to
motivate users to understand the subject of HDCT and thereby enable
its wider dissemination. Less mathematical users will appreciate the
computational support we will develop to underpin HDCT. Indeed, our
overall aim is to ensure that HDCT becomes as as indispensible as
category theory is becoming. The potential impact of this research can
also be assessed by it's adventure, timeliness and novelty which we
now discuss.

{\em Adventure:} In order to ensure that HDCT matures into a
fundamental tool capable of use by those already using category
theory, we have designed an adventurous proposal which concentrates on
establishing fundamental results which are likely to stand the test of
time, and on exploiting these results by applying them to current
problems in the theory of computation. This is evidenced by the scope
of the project which ranges from fundamental but theoretical questions
to the production of applications and even associated code.

This is clearly not incremental research. Prof Ghani has not worked on
HDCT before while Dr Cheng has not worked on programming languages
research before. They are therefore ideal collaborators as they share
a common categorical language to allow them to communicate, but bring
expertise which the other lacks but is interested in learning about.
Their different perspectives is likely to stimulate new thinking
about, and applications of, HDCT. Further, previous research on HDCT
has tended to remain within theoretical domain. The central thrust of
this proposal - to turn HDCT into a widely used technique - is thus a
new direction for HDCT.

{\em Timeliness:} This is an excellent moment to undertake this
research. The advent of dependently typed programming languages gives
new ways for non experts to understand HDCT via an implemntation of
it. In a different direction, new advances in HDTT will help
researchers in HDCT while progress we make in HDCT can feed into HDTT. 

{\em Novelty:} This is novel research. We intend not to confine
ourselves to either the theoretical domain or simply concentrate on
applications. Rather, for us, there is a symbiotic relationship
between mathematics, programming and the design of programming
languages, and any attempt to sever this connection will diminish each
component. At the centre of this relationship is the desire to use
mathematics to understand the nature of computation, and then to
reflect that understanding in programming languages. Hence our
proposal is novel in taking cutting edge mathematics and applying it
to make cutting edge programming techniques.

{\bf Dissemination:} We have a track record of submitting papers to
internationally leading journals and conferences and shall use them as
the principal vehicle for dissemination of our results. We are also
active participants in more informal meetings such as the {\em British
  Colloquium on Theoretical Computer Science}. A web-site will detail
the state of the project. We are active members of relevant email
lists and will also use them to disseminate our results. We will
collaborate extensively with our project partners which will further
aid dissemination. Finally, dissemination will be enhanced by our
planned twice yearly project meetings (see below) which will be open
to other researchers to attend and contribute to.

{\bf Management, Planning and Coordination:} We have carefully planned
this project. Finally, while many of the
objectives will influence others, none are prerequisites. Hence lack of
progress on one objective will not prohibit progress on other
objectives, eg ... 

{\em Coordination:} During the first three years, we have planned
twice yearly intensive meetings rotating between the sites. These
meetings will be open so that interested academics may attend giving
further opportunities for input, feedback and dissemination. During
that period, we also plan one individual visit between sites per
person per year for more focussed work on specific topics.  A wiki and
blog will be used for interaction between participants and to create a
record of progress. 

{\em Feasability:} We are the right people to execute this
research. Dr Cheng is one of the world leaders on HDCT while Prof
Ghani is an expert in using categorical abstractions to help structure
and reason about computational processes.

Further, the Universities of Strathclyde and Sheffield all have
research groups dedicated to understanding category theory and the
mathematical foundations of computation. Strathclyde is also
interested in developing the programming languages of the future and
this research will feed into that goal. Our contacts (detailed under
dissemination) further enhance the feasability of this project.



\subsection*{2.4 \gap Methodology and Research  Programme}\label{sec:detailed}

\subsection*{2.4.1 \gap Theoretical Research} 
%
We have presented the motivation for higher dimensional category
theory - now let us say exactly what we will do to bring this
potential to fruition.

{\bf WP1. Higher Dimensional Categories}
%
Both theory and applications of higher category theory have been
plagued to date by the complexity of the elementary notions. 
Just of weak $\omega$-category there are many alternative
definitions whose formulations are non-elementary and which
are, moreover, not obviously equivalent. This is not in principle a
problem; the existence of many alternative but equivalent versions of
the same concept is a testament to its fundamental nature.  However,
and more importantly, the payload of doing higher category theory is,
at the moment, so high that it strangles its further development.

\emph{We believe that this is not necessary.} Although an
$\omega$-category is undeniably a complicated notion requiring
nontrivial abstractions it is an appropriate language to deal with the
complexity in derived terms that is missing. We are proposing to
develop such a language and we believe (Martin L\"of) Type Theory is
the ideal logical framework for this task. Recent research shows deep
connections between Type Theory and higher category theory which
indicates that Type Theory isn't just an general tool for the
development of various logics but in this case also the
canonical (internal, inherent) one.

\emph{We will thus develop a language for higher category theory}
based on the so-called opetopic approach and iterated indexed
containers.  Although this seems straightforward enough now, things
become interesting once extensional equality is dropped\footnote{We
  have to do without extensionality so we are not circular} and notions
previously equal become distinct. For instance, only of indexed
containers there are four varieties, depending on whether we work with
indexed or fibered families.

\emph{We will adhere to the a syntactical approach to the definition.}
Explicitly, we define the syntax of an $\omega$-category using
indexed containers (as opetopes), and then define an $\omega$-category
as any opetopic set with interpretation of the syntax. This approach
has proved vital in our previous work \cite{} in circumventing the
usual problem with coherence [MacLane] arising in \textit{non-faithful
models}\marginpar{What is \\ this?}. It also allowed us to eliminate equational axioms and
quotienting, by careful indexing, which is otherwise necessary in the
usual semantical approaches.

\emph{To establish relevance to the existing definitions}, we will
continue the ongoing program of unifying foundations of higher
category theory and show our approach equivalent to the major other
approaches.


{\bf WP2. Higher Dimensional Ontology}
%
Cells and categories are the letters and alphabets in our language for
higher categories. Once we have defined them, we start building the
basic vocabulary of words, i.e. the various universal
constructions. The notion of universality and identification of the
key universal constructions are the most important
contributions of ordinary category theory to modern mathematics and
computing science.  We will develop their higher-dimensional versions.

\emph{As with the notion of a category itself, the fundamental
  obstacle to a naive approach would be the aforementioned
  combinatorial complexity.}  To tackle it we need a good mathematical
calculus. Therefore we start by developing a calculus of opetopes
based on the mathematics of indexed containers \cite{?,?,?}.
%
% At the end of \cite{}, such a
% calculus of opetopes is started in terms of the diagrammatic
% representation of opetopes presented in the paper. 
%
The starting points should be opetopic functors and
their transformations. Opetopic functors have appeared in the literature
before \cite{}.  Naturally they are defined as maps of the underlying
opetopic sets \cite{?}, either preserving or not the universality of
fillers. If preservation is omitted, the resulting notion of a functor
is \emph{lax}, otherwise we obtain a \emph{strong} notion from
uniqueness of universal constructions up to isomorphism/weak equivalence.
%
\textbf{oxr: are strict functors interesting?}
%
For a pair of functors $F$, $G$ from $\mathcal{C}$ to $\mathcal{D}$, a
transformation from $F$ to $G$ should be collection of \emph{opetope
  morphisms} from $F(o)$ to $G(o)$ for each opetope in
$\mathcal{C}$. The missing ingredient, \emph{opetope morphism}, will
be defined inductively on the dimension of the opetope. Our paper
\cite{} develops a roughly analogical notion of a morphism in the
globular setting. Once we have developed the mathematics of functors
and their transformations, we derive the notions of adjunctions, Kan
extensions and from there the various limits and colimits.

Finally, we compare our functors to notions of
functors developed for other notions of $\omega$-categories, e.g.
Garner's \emph{homomorphisms} \cite{} for Batanin's
$\omega$-categories. 

% Proposals for functors, or so-called , have appeared in
% the literature before. Garner \cite{} develops a notion of a
% homomorphism of Batanin's globular $\omega$-categories. For opetopic categories,


% If we assume for a moment that there is (something like) an
% $\omega$-category of $\omega$-categories, where objects (0-opetopes)
% are categories and arrows between objects (1-opetopes) are functors,
% then \emph{transformations} must be higher opetopes. We intentionally
% don't say \emph{natural} transformations as the usual notion of
% naturality will need to be relaxed in the weak $\omega$-setting as is
% the case already in the low dimensions. The correct notion(s) of
% naturality should also determine the notion of universal construction
% (limit, colimit, etc. ) under consideration. 

% A notion of \emph{universality} is built into opetopic categories. We
% will look to reuse that built in notion and define 
% Kan extension as a filler of an appropriate niche. 


% \textbf{oxr:} that all sounds like a lot of waffle. 
%previous work



% \emph{We are going to define and investigate the different notions of
%   functors, associated notion of transformations and universality.}
% Next we define a notion of Kan Extension, from which all other
% universal constructions follow by instantiation following the
% principle of generalisation of lower-dimensional examples. 

{\bf WP3: The category of $\omega$-categories}
\begin{itemize}
\item Is it a problem  / possible at all / has it been investigated ? 
\item If it's not possible, we need constructions on categories
  (products, units, etc.) 
\item If it's been done, we would just take WP2 and apply.
\item If it's not known, we want to do it. 
\item Let's ask Eugenia !
\end{itemize}


{\bf WP4: HDTT}
\begin{itemize}
\item Martin L\"of Type Theory but without UIP.
\end{itemize}
Connections of Martin L\"of Type Theory and higher category theory
($\omega$-groupoids) have come to light with the work of Gambino,
Garner, Lumsdaine, van den Berg and others \cite{} who have shown
that without Uniqueness of Identity
Proofs principle (UIP) identity types have the structure of
$\omega$-groupoid ($(\infty,0)$-category). While being extremely
useful and theoretically a relatively manageable object, turning Type
Theory without UIP into a useful programming and theorem proving
paradigm is a difficult open problem. 

% bullshit: 
% Deepening of our understanding of
% the categorical side of things will contribute greatly to this
% end. 

% The opetopic definition of $\omega$-category is ideal for
% formalisation in type theory because it lacks equational axioms -- it
% postulates data from data without equations, as all equations have
% been turned into more data. We would like to provide a definition of a
% weak $\omega$-category internal to type theory.
% %
% Such definition could form the basis for an axiomless model of
% univalent Higher Dimensional Type Theory where UIP doesn't hold. This
% could provide a platform for further development of programming
% languages [Epigram]. 

\begin{itemize}
\item Functional extensionality
\end{itemize}


One important open problem where ordinary category
theory has already celebrated a great success is understanding of
(ordinary) inductive types via the so-called \emph{initial algebra
  semantics}. The higher-dimensional formulation of inductive types is
an ongoing work in progress \cite{} with exciting
applications\cite{}. We plan to tackle the categorical semantics in
parallel, with the aim of both driving the \emph{correct} formulation
and providing a clear semantics and theoretical background.




%%% Local Variables: 
%%% mode: latex
%%% TeX-master: "hdct"
%%% End: 


\vspace*{0.02in}




\small

%\setlength{\bibsep}{0pt}
\bibliographystyle{plain}
\bibliography{hdct}
% \begin{thebibliography}{}

% \vspace*{-0.2in}

% \setlength{\parindent}{0pt}
% \setlength{\columnsep}{0.3in}
% \begin{multicols}{2}

% \bibitem[AAG05]{aag05}
% M. Abott, T. Altenkirch, and N. Ghani. Containers - Constructing
% strictly positive types. {\em TCS} 342 (2005), 3--27.


% \bibitem[Bar92]{bar92}
% H. Barendregt.  Lambda calculi with types. {\em Handbook of Logic in
%   Computer Science}, 1992. 


% \bibitem[Jac99]{jac99} B. Jacobs. {\em Categorical Logic and Type
% Theory}. North Holland, 1999.

% \bibitem[MM04]{mm04}
% C. McBride and J. McKinna. The view from the left.
% {\em JFP} 14(1) (2004), 69-111.

% \bibitem[MTHM97]{mthm97}
% R. Milner, M. Tofte, R. Harper, and D. MacQueen. {\em The Definition
%   of Standard ML, Revised Edition}. MIT Press, 1997.

% \bibitem[Mog91]{mog91}
% E. Moggi. Notions of computation and monads.
% {\em Inf. and Comp.} 93(1) (1991), 55-92.


% \bibitem[PJ03]{pj03}
% S.~L.\ {Peyton Jones}. {\em Haskell 98 Language and Libraries: {T}he
% Revised Report}. Cambridge, 2003.



% \end{multicols}
% \end{thebibliography}

\pagebreak

%\section*{Diagrammatic Workplan}

%\epsfbox{indexed.eps}  % inputs given file

\pagebreak

\end{document}




\subsection*{3. \gap Additional Considerations}

\begin{itemize}

\item {\bf Why us?}\\

\item {\bf Success Criteria.}\\
 
\item {\bf Relevance to Beneficiaries and Related Work.}

\item {\bf Impact.} 

\item {\em ...Adventurous:} 

\item {\em ...Timely:} 

\item {\em ...Novel:} 

\item {\bf Feasibility.} 

\item {\bf Transformative:}

\item {\bf EPSRC Politics:}

\item {\bf Project Partners:}