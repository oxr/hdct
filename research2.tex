\subsection*{2.4 \gap Methodology and Research  Programme}\label{sec:detailed}

\subsection*{2.4.1 \gap Theoretical Research} 
%
We have presented the motivation for higher dimensional category
theory - now let us say exactly what we will do to bring this
potential to fruition.

{\bf WP1. Higher Dimensional Categories}
%
Both theory and applications of higher category theory have been
plagued to date by the complexity of the elementary notions. 
Just of weak $\omega$-category there are many alternative
definitions whose formulations are non-elementary and which
are, moreover, not obviously equivalent. This is not in principle a
problem; the existence of many alternative but equivalent versions of
the same concept is a testament to its fundamental nature.  However,
and more importantly, the payload of doing higher category theory is,
at the moment, so high that it strangles its further development.

\emph{We believe that this is not necessary.} Although an
$\omega$-category is undeniably a complicated notion requiring
nontrivial abstractions it is an appropriate language to deal with the
complexity in derived terms that is missing. We are proposing to
develop such a language and we believe (Martin L\"of) Type Theory is
the ideal logical framework for this task. Recent research shows deep
connections between Type Theory and higher category theory which
indicates that Type Theory isn't just an general tool for the
development of various logics but in this case also the
canonical (internal, inherent) one.

\emph{We will thus develop a language for higher category theory}
based on the so-called opetopic approach and iterated indexed
containers.  Although this seems straightforward enough now, things
become interesting once extensional equality is dropped\footnote{We
  have to do without extensionality so we are not circular} and notions
previously equal become distinct. For instance, only of indexed
containers there are four varieties, depending on whether we work with
indexed or fibered families.

\emph{We will adhere to the a syntactical approach to the definition.}
Explicitly, we define the syntax of an $\omega$-category using
indexed containers (as opetopes), and then define an $\omega$-category
as any opetopic set with interpretation of the syntax. This approach
has proved vital in our previous work \cite{} in circumventing the
usual problem with coherence [MacLane] arising in \textit{non-faithful
models}\marginpar{What is \\ this?}. It also allowed us to eliminate equational axioms and
quotienting, by careful indexing, which is otherwise necessary in the
usual semantical approaches.

\emph{To establish relevance to the existing definitions}, we will
continue the ongoing program of unifying foundations of higher
category theory and show our approach equivalent to the major other
approaches.


{\bf WP2. Higher Dimensional Ontology}
%
Cells and categories are the letters and alphabets in our language for
higher categories. Once we have defined them, we start building the
basic vocabulary of words, i.e. the various universal
constructions. The notion of universality and identification of the
key universal constructions are the most important
contributions of ordinary category theory to modern mathematics and
computing science.  We will develop their higher-dimensional versions.

\emph{As with the notion of a category itself, the fundamental
  obstacle to a naive approach would be the aforementioned
  combinatorial complexity.}  To tackle it we need a good mathematical
calculus. Therefore we start by developing a calculus of opetopes
based on the mathematics of indexed containers \cite{?,?,?}.
%
% At the end of \cite{}, such a
% calculus of opetopes is started in terms of the diagrammatic
% representation of opetopes presented in the paper. 
%
The starting points should be opetopic functors and
their transformations. Opetopic functors have appeared in the literature
before \cite{}.  Naturally they are defined as maps of the underlying
opetopic sets \cite{?}, either preserving or not the universality of
fillers. If preservation is omitted, the resulting notion of a functor
is \emph{lax}, otherwise we obtain a \emph{strong} notion from
uniqueness of universal constructions up to isomorphism/weak equivalence.
%
\textbf{oxr: are strict functors interesting?}
%
For a pair of functors $F$, $G$ from $\mathcal{C}$ to $\mathcal{D}$, a
transformation from $F$ to $G$ should be collection of \emph{opetope
  morphisms} from $F(o)$ to $G(o)$ for each opetope in
$\mathcal{C}$. The missing ingredient, \emph{opetope morphism}, will
be defined inductively on the dimension of the opetope. Our paper
\cite{} develops a roughly analogical notion of a morphism in the
globular setting. Once we have developed the mathematics of functors
and their transformations, we derive the notions of adjunctions, Kan
extensions and from there the various limits and colimits.

Finally, we compare our functors to notions of
functors developed for other notions of $\omega$-categories, e.g.
Garner's \emph{homomorphisms} \cite{} for Batanin's
$\omega$-categories. 

% Proposals for functors, or so-called , have appeared in
% the literature before. Garner \cite{} develops a notion of a
% homomorphism of Batanin's globular $\omega$-categories. For opetopic categories,


% If we assume for a moment that there is (something like) an
% $\omega$-category of $\omega$-categories, where objects (0-opetopes)
% are categories and arrows between objects (1-opetopes) are functors,
% then \emph{transformations} must be higher opetopes. We intentionally
% don't say \emph{natural} transformations as the usual notion of
% naturality will need to be relaxed in the weak $\omega$-setting as is
% the case already in the low dimensions. The correct notion(s) of
% naturality should also determine the notion of universal construction
% (limit, colimit, etc. ) under consideration. 

% A notion of \emph{universality} is built into opetopic categories. We
% will look to reuse that built in notion and define 
% Kan extension as a filler of an appropriate niche. 


% \textbf{oxr:} that all sounds like a lot of waffle. 
%previous work



% \emph{We are going to define and investigate the different notions of
%   functors, associated notion of transformations and universality.}
% Next we define a notion of Kan Extension, from which all other
% universal constructions follow by instantiation following the
% principle of generalisation of lower-dimensional examples. 

{\bf WP3: The category of $\omega$-categories}
\begin{itemize}
\item Is it a problem  / possible at all / has it been investigated ? 
\item If it's not possible, we need constructions on categories
  (products, units, etc.) 
\item If it's been done, we would just take WP2 and apply.
\item If it's not known, we want to do it. 
\item Let's ask Eugenia !
\end{itemize}


{\bf WP4: HDTT}
\begin{itemize}
\item Martin L\"of Type Theory but without UIP.
\end{itemize}
Connections of Martin L\"of Type Theory and higher category theory
($\omega$-groupoids) have come to light with the work of Gambino,
Garner, Lumsdaine, van den Berg and others \cite{} who have shown
that without Uniqueness of Identity
Proofs principle (UIP) identity types have the structure of
$\omega$-groupoid ($(\infty,0)$-category). While being extremely
useful and theoretically a relatively manageable object, turning Type
Theory without UIP into a useful programming and theorem proving
paradigm is a difficult open problem. 

% bullshit: 
% Deepening of our understanding of
% the categorical side of things will contribute greatly to this
% end. 

% The opetopic definition of $\omega$-category is ideal for
% formalisation in type theory because it lacks equational axioms -- it
% postulates data from data without equations, as all equations have
% been turned into more data. We would like to provide a definition of a
% weak $\omega$-category internal to type theory.
% %
% Such definition could form the basis for an axiomless model of
% univalent Higher Dimensional Type Theory where UIP doesn't hold. This
% could provide a platform for further development of programming
% languages [Epigram]. 

\begin{itemize}
\item Functional extensionality
\end{itemize}


One important open problem where ordinary category
theory has already celebrated a great success is understanding of
(ordinary) inductive types via the so-called \emph{initial algebra
  semantics}. The higher-dimensional formulation of inductive types is
an ongoing work in progress \cite{} with exciting
applications\cite{}. We plan to tackle the categorical semantics in
parallel, with the aim of both driving the \emph{correct} formulation
and providing a clear semantics and theoretical background.




%%% Local Variables: 
%%% mode: latex
%%% TeX-master: "hdct"
%%% End: 
